
On the seventh day. The machines became aware.
\begin{tabular}{l}
OBTAIN\\
\texttt{lynx -dump url} get text of page\\
\texttt{curl url/apicall}\\
\texttt{wc } word count\\
\\
SCRUB\\
print cols 1 and 2\\
\texttt{awk '\{print \$1\$2\}' logs.txt}\\
print whole lines\\
\texttt{awk '\{print \$0\}' file}\\
print \texttt{linenumber col1 -> numfields-2}\\
\texttt{awk '\{print NR,$1 " -> " $(NF-2)\}'}\\
use colon delimiter\\
\texttt{awk 'BEGIN\{FS=":"\}\{print \$1\}'}\\
print if second last field is '200'\\
\texttt{awk '\{if (\$(NF-2) == "200") \{print \$0\}\}'}\\
\\
EXPLORE\\
print some random lines\\
\texttt{cat file | sort -R | head -n 10}\\
\texttt{less file} \emph{remember Vim cmds}\\
MODEL\\
\\
INTERPRET\\
\\ \end{tabular} 
\begin{tabular}{l}
Python Tricks\\
\texttt{if "str" in string:}\\
\texttt{\ \ \  print "str is in string"}\\
Pretty print a list\\
\texttt{print ",".join(list\_of\_things)}\\
Filtering lists:\\
\texttt{numsUnder4 = filter(lambda x: x < 4,nums)}\\
is equivalent to\\
\texttt{numsUnder4 = [n for n in nums if n < 4]}\\
Generator Expressions are more efficient:\\
\texttt{squaresU10 = (n*n for n in nums if n*n < 10)}\\
Each successive value of squaresU10 can be gotten by calling .next(), or\ldots\\
\texttt{for s in squaresU10:}\\
\texttt{\ \ \  print s}\\
Reducing:\\
\texttt{result = reduce(lambda a,b: a*b, numbers)}\\

\end{tabular}
%TODO command line tools
%http://www.pixelbeat.org/cmdline.html
\vfill
Stand back. I know regular expressions.
% http://www.addedbytes.com/cheat-sheets/regular-expressions-cheat-sheet/
\bigskip
\begin{tabular}{ll}
\texttt{([A-Za-z0-9-]+)}&letters, numbers and hyphens\\
\end{tabular}
\vfill
``All models are wrong, some are useful.'' - George Box
\vfill
\clearpage

