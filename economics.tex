
\subsection*{Financial Markets}
Types of markets: \emph{equity/stock, bond, money/credit, commodity, physical
asset, FOREX, derivatives}.

Risk averse preferences under uncertainty == MU falling as wealth increases...
risk averse/neutral/loving leads to indifference curves.

An Edgworth box traces the outline of all possible allocations of 2 assets. An
allocation of the endowment that improves the welfare of a consumer without
reducing the welfare of another is a Pareto-improving allocation. Since
consumers can refuse to trade the only outcomes from trade are pareto
improving. 

\begin{description}
\item[risk] probabilities known
\item[uncertainty] probabilities unknown
\end{description}
The \emph{expected utility hypothesis} implies that probs are applied to each
state and that the decision maker maximises the expected utility...
\[E[u(W)]=U(W_1,...W_l)=\pi_1u(W_1)+...\pi_lu(W_l)\]
\subsubsection*{Risk aversion (RA)}
\[
\begin{array}{rl}
u''(W) < 0 &\mbox{averse}\\
u''(W) = 0 &\mbox{neutral}\\
u''(W) > 0 &\mbox{loving}\\
-W\frac{u''(W)}{u'(W)} &\mbox{index of rel RA}\\
\frac{-u''(W)}{u'(W)} &\mbox{index of abs RA}\\
\mbox{CRRA }u(W) =  &W^{1-\gamma}/(1-\gamma) \mbox{ if } \gamma \ne 1\\
                    &lnW \mbox{ if } \gamma = 1\\
\mbox{CARA }u(W) =& 1-e^{\phi W} \phi > 0\\
\mbox{quadratic }u(W) =&W - bW^2, b>0\\
\end{array}
\]

\subsubsection*{Portfolio Selection in EUH...}

\begin{tabular}{rl}
&\\
\emph{Some definitions...}&\\
budget constraint& $p_1x_1+..+p_nx_n=A$\\
A &initial wealth\\
$x_j$, $p_j$&holding, price of asset $j$\\
$a_j=p_jx_j/A$& proportion invested in $j$\\
terminal wealth& $W = (1 + r_p)A$\\
$r_p$& is rate of return  \\
$v_j=(1+r_j)p_j$ & payoff of asset $j$\\
$x_0$&holding of risk free asset\\
$v_{kj}$& arrow security pays 1 if\\
&$k=j$ and 0 otherwise\\
$r_0$&risk free asset\\
\end{tabular}
\begin{tabular}{rl}
$var[W]$&$=E[W-E[W]]^2i$\\
        &$E[W^2]-[E[W]]^2$\\
$\mu_j=E[r_j]$&$\mu_P=E[r_P]$\\
$\sigma_{jP}$&$=cov[r_j,r_P]$\\
$\sigma_j=+\sqrt{\sigma_{jj}}$&stdev of $j$\\
$\rho_{ij}=\sigma_{ij}/(\sigma_i\sigma_j)$& correlation coefficient\\
$a_0,a_1,a_2,\ldots,a_n$&portfolio\\
$\mu_P=\sum^n_{j=0}a_j\mu_j$&return of portfolio\\
$\sigma^2_P=\sum^n_{i=0}\sum^n_{j=0}a_ia_j\sigma_{ij}$&var of portfolio\\
$\sigma_{jP}=\sum^n_{i=0}a_i\sigma_{ij}$&covar between$j$ and $P$\\
$s_j=\frac{\mu_j-r_0}{\sigma_j}$& Sharpe ratio\\
&\\
\end{tabular}

An \emph{Arbitrage Portfolio} is where $A = 0$

The \emph{Fundamental Valuation Relationship} is a collection of first order
conditions, one for each asset. In general form: \[E[(1+r_j)H]=1 \mbox{for all
j}\]

H is a random variable specific to the context. It varies across states and
agents, reflecting agents' holdings of portfolios.


\[\mathcal{L}=\pi_1u(W_1)+\pi_2u(W_2)+...+\pi_lu(W_l)+\]
\[\lambda(A-p_0x_0-p_1x_1-p_2x_2-..-p_nx_n)\]
where
\[W_k=v_0x_0+v_{k1}x_1+v_{k2}x_2+...+v_{kn}x_n\]
The fundamental Valuation Relationship for (single period) portfolio allocation
problem is:

\[E[(1+r_j)u'(W)]=\lambda \mbox{ for all } j\]
\[H_k=u'(W_k)/\lambda\]
\subsubsection*{Implications of complete asset markets}
\emph{If} agents agree on probs and asset markets are complete then $H$ becomes
common ($H_k$ may still differ across agents as $u(W)$ differs). We need arrow
securities or to construct portfolios that pay in only one state.

\subsubsection*{Risk Neutrality}
FVR in presence of risk free asset
\[E[(r_j-r_0)u'(W)]=0\]
Risk neutrality $u''(W)=0$ means $u'(W)=c$, substituting these into the FVR we
get $E[r_j]=r_0$ << this is an identity. If it does not hold an investor will
borrow at $r_0$ and invest in an asset with the highest $E[r_j]$

So optimization of portfolios for risk-neutral agents is a moot question.

\subsubsection*{Mean-Variance Model}
Assume $u(W)=W-bW^2$ then
\[E[u(W)]=F(E(W),var[w])\]
using variance defs we get 
\[E[u(W)]=E[W]-b[var[W]+(E[W])^2]\]
variance of the portfolio
\[\sigma^2_P=E[(r_P-\mu_P)^2]\]
FVR in quadratic $u$ case
\[\frac{\mu_j-r_o}{\sigma_{jP}/\sigma{P}}=\frac{\mu_P-r_0}{\sigma_P}\mbox{ for
all } j\]

So $\sigma_{jP}/\sigma_P$ is the increment to risk if we hold a bit more of
$j$.

FVR states that the expected return to any asset ( $\mu_j - r_0$) has to be
proportional to its contribution to overall risk $\sigma_{jP}/\sigma{P}$.

Investor acts to maximize $G(\mu_P,\sigma^2_P)$ the shape of G depends on the
utility function. 

$G=\mu_P-\alpha \sigma^2_P$ is a popular example (if all asset returns are
normally dist and 

$u=1-e^{\phi W}$) then $G$ is equivalent to expected utility maximisation and
$\alpha=\frac{\phi}{2}$. 

\subsubsection*{Intertemporal choice and the equity premium puzzle}

Consideer two periods where  all wealth is consumed at $t+1$.

\begin{tabular}{rl}
$C_t, C_{t+1}$& consumption\\
$W_t, W_t-C_t$& endowments\\
$C_{t+1}=(1+r_{t+1})(W_t-C_t)$& IT budget constriaint\\
$U(C_t,C_{t+1})=$&$u(C_t)+\delta u(C_{t+1})$\\
$\delta$& subjective disc factor\\
$(1/\delta)-1$& rate of time prefs\\
$u'>0$&$u''<0$\\
\end{tabular}

Uncertainty: Recall FVR, if EUH holds:

\[H=\delta\frac{u'(C_{t+1})}{u'(C_t)}\]
\begin{description}
\item[Lifetime portfolio selection] conventional wisdom is to take more risk
when young. Not according to this model.

\item[abandoning the EUH] we can consider a \emph{target wealth} or
\emph{minimizing the risk of loss} model.

\item[The role of human capital] if labour income is risky (risk of redundancy)
then less risk is taken with portfolio. Can argue that young people, secure in
their jobs should invest in equities. Older people have lower humancapital and
should invest in bonds.

\item[The Equity Premium Puzzle] Investors tend to favour low-risk bonds to
their detriment. The value of $\gamma$ needed to justify the premium is around
8.5 whereas most studies in other contexts found $\gamma$ to be less than 3.

\item[risk free rate puzzle] investors transfer too much wealth from the
present to the future for plausible values of $\delta$

\end{description}

What to do... MEASURE HOW PEOPLE BEHAVE \emph{ - David Griffin}

\subsubsection*{The Mean Variance $(\mu,\sigma)$ Approach}
Plane is $\mu_P,\sigma_P)$. Draw triangle between asset 1, asset 2 and V on
vert ($\mu_P$) axis. V is the risk free portfolio. Portfolios can exist inside
triangle and outside triangle if we allow short selling. If there is no risk
free asset we start with two assets, combine them and continue.

\emph{First mutual fund theorem} of portfolio analysis: given $n>2$ portfolios,
there exist two composite assets (mutual funds) such that evry
$(\mu_P,\sigma_P)$ on the efficient frontier can be obtained holding only the
two fuds.

$r_0$ and $r_l$ if it is different are on the vertical axis. Plot line from
them tangent to frontier.

\emph{second mutual fund theorem} any efficient portfolio can be obtained by
holding risk free asset and a mutual fund.

All efficient portfolios share the same Sharpe ratio.

%\subsubsection*{The CAPM}
%Assumptions: \emph{equilibrium, frictionless, unlimited borrowing/lending,
%assets are divisible, investors are price takers, taxes are neutral} We use
%mean-variance portfolio selection, and homogenous beliefs.
%
%\[\beta_{jZ}=\sigma_{jZ}/\sigma^2_Z\mbox{ \ }z_1,\ldots,z_n \mbox{ is
%portfolio} \sum z_j=1\]
%
%CML goes from $r_0$ on $\mu$ axis with slope of $(\mu_M-r_0)/\sigma_M$
%
%Characteristic line goes through origin of $\mu_j-r_0,\mu_M-r_0)$ plane with
%slope $\beta_j$
%
%SML goes from $r_0$ to $(\mu_M,1)$ on $(\mu_j,\beta_j)$ plane.
%
%Assets above the SML are underpriced.
\subsubsection*{Options}
\begin{description}
\item[call] the right to purchase at the exercise or strike price (writer must
sell)
\item[put] the right to sell at ex or strike price (writer must buy)
\item[American] can be exercised on or before the expiry date
\item[European] can be exercised only at the expiry date
\end{description}

\clearpage

